\documentclass{scrartcl}
\usepackage[ngerman]{babel}
\usepackage{rwukoma}
\usepackage[hidelinks]{hyperref}
\usepackage{lipsum,caption}
\usepackage{graphicx}
\usepackage{listings}
\usepackage{color}
\usepackage{caption}
\usepackage{xparse}
\usepackage{quoting}
\usepackage{inputenc}
\usepackage{csquotes}
\usepackage{acronym}
\usepackage{hyperref}
\usepackage{lmodern}
\usepackage{array}
\usepackage{subfiles}
\usepackage{amssymb}
\usepackage{subcaption}
\usepackage{amsmath}
\usepackage{float}
\usepackage{scrhack}
\usepackage{microtype}
\usepackage{svg}
\usepackage{pgfgantt}
\usepackage{pgfplots}
\usepackage{lmodern}
\usepackage{tikz}
\usepackage[acronym,nonumberlist]{glossaries}
\usetikzlibrary{shapes,arrows,positioning, decorations.pathmorphing}
\usetikzlibrary{calc}
\usepackage[backend=biber,
style=apa,
]{biblatex}
\addbibresource{literatur.bib}

\renewcaptionname{ngerman}{\tablename}{Tabelle}
\renewcaptionname{ngerman}{\figurename}{Abbildung}
\renewcaptionname{ngerman}{\listtablename}{Tabellenverzeichnis}
\renewcaptionname{ngerman}{\listfigurename}{Abbildungsverzeichnis}
\renewcommand{\lstlistingname}{Programmcode}
\renewcommand{\lstlistlistingname}{Programmcode-Verzeichnis}
\addto\extrasngerman{%
\def\subsectionautorefname{Abschnitt}%
\def\subsubsectionautorefname{Abschnitt}%
}


\DeclareCaptionType{equ}[Formel][Formelverzeichnis]
\DeclareCaptionType{code}[Code][Codeverzeichnis]

% Codestyle preferences

\definecolor{codegreen}{rgb}{0,0.6,0}
\definecolor{codegray}{rgb}{0.5,0.5,0.5}
\definecolor{codepurple}{rgb}{0.58,0,0.82}
\definecolor{backcolour}{rgb}{0.95,0.95,0.92}

% Codestyle
\lstdefinestyle{mystyle}{
    backgroundcolor=\color{backcolour},   
    commentstyle=\color{codegreen},
    keywordstyle=\color{magenta},
    numberstyle=\tiny\color{codegray},
    stringstyle=\color{codepurple},
    basicstyle=\ttfamily\footnotesize,
    breakatwhitespace=false,         
    breaklines=true,                 
    captionpos=b,                    
    keepspaces=true,                 
    numbers=left,                    
    numbersep=5pt,                  
    showspaces=false,                
    showstringspaces=false,
    showtabs=false,                  
    tabsize=2
}


% Define block styles
\tikzset{
	pap/.style={
		draw,
		align=flush center,
		on grid
	},
	paprectangle/.style={
		pap,
		text width=3cm-2*\pgfkeysvalueof{/pgf/inner xsep},
		inner sep=4pt
	},
	papStart/.style = {
			paprectangle,
			rounded corners=10pt,
	},
	papEnd/.style = {papStart},
	papProcess/.style = {paprectangle},
	papLine/.style = {
			draw,
			-{Stealth[length=9pt,width=6pt]},
		},
	papLabel/.style={
		label distance=.5cm,
		font=\footnotesize\itshape
	},
}
\tikzset{line/.style={draw, -latex}}
\lstset{style=mystyle}
\tikzset{Shift/.style 2 args={shift={(+#1.5\pgflinewidth,+#2.5\pgflinewidth)}}}
\newcommand*{\pgfkeysstylelet}[2]{\expandafter\let\csname pgfk@#1/.@cmd\expandafter\endcsname\csname pgfk@#2/.@cmd\endcsname}
\ganttset{
    /tikz/broken/.style={
        decoration={
            name=zigzag,
            amplitude=+1pt,
            segment length=+2pt}},
    broken right/.code={
        \pgfkeysstylelet{/pgfgantt/bar backup}{/pgfgantt/bar}
        \pgfkeysalso{
            /pgfgantt/bar left shift=-.5\pgflinewidth/\ganttvalueof{x unit},
            /pgfgantt/bar right shift=(.5\pgflinewidth+.5\pgfdecorationsegmentamplitude)/\ganttvalueof{x unit},
            /pgfgantt/bar/.append style={
                draw=none,
                path picture={
                    \draw[/pgfgantt/bar backup, broken] ([Shift=+-] path picture bounding box.north west) |-
                                                        ([Shift=-+, xshift=-\pgfdecorationsegmentamplitude-.6\pgflinewidth] path picture bounding box.south east)
                                           decorate {-- ([Shift=--, xshift=-\pgfdecorationsegmentamplitude-.6\pgflinewidth] path picture bounding box.north east)}
                                                     -- cycle;}}}},
    broken left/.code={
        \pgfkeysstylelet{/pgfgantt/bar backup}{/pgfgantt/bar}
        \pgfkeysalso{
            /pgfgantt/bar right shift=.5\pgflinewidth/\ganttvalueof{x unit},
            /pgfgantt/bar left shift=(-.5\pgflinewidth-.5\pgfdecorationsegmentamplitude)/\ganttvalueof{x unit},
            /pgfgantt/bar/.append style={
                draw=none,
                path picture={
                    \draw[/pgfgantt/bar backup, broken] ([Shift=--] path picture bounding box.north east) |-
                                                        ([Shift=++, xshift=\pgfdecorationsegmentamplitude+.6\pgflinewidth] path picture bounding box.south west)
                                           decorate {-- ([Shift=+-, xshift=\pgfdecorationsegmentamplitude+.6\pgflinewidth] path picture bounding box.north west)}
                                                     -- cycle;}}}}}



\begin{document}
\pagenumbering{Roman}
	\begin{titlepage}
		\subfile{sections/Titelseite.tex}
	\end{titlepage}
	\clearpage 
	\setcounter{page}{1}
	\section{Eidesstattliche Erklärung}
		\subfile{sections/Eidesstattliche_Erklaerung.tex}
		\clearpage
	\section{Aufgabenbeschreibung}
		\subfile{sections/Aufgabenbeschreibung.tex}
		\clearpage
	\tableofcontents
	\clearpage
	\listoffigures
	\listoftables
	\listof{equ}{Formelverzeichnis}
	\clearpage
	\lstlistoflistings
	\clearpage
	\section{Abkürzungen}
		\subfile{sections/Abkuerzungen.tex}
		\clearpage
	\section{Abstract}
		\subfile{sections/Abstract.tex}
	\clearpage
\pagenumbering{arabic}
\setcounter{page}{1}
	\section{Grundlagen}
		In diesem Kapitel werden die Grundlagen dieser Arbeit zum aktuellen Stand der Technik vorgestellt.
		Dabei werden die verwendeten Systeme sowie auch benötigte Protokolle und Algorithmen erklärt.
		\subsection{Echtzeitsysteme} \label{section:echtzeitsysteme}
			\subfile{sections/Echtzeitsysteme.tex}
		\subsection{Scheduler} \label{section:scheduler}
			\subfile{sections/Scheduler.tex}
		\subsection{FreeRTOS} \label{section:freertos}
			\subfile{sections/FreeRTOS.tex}
		\subsection{Periodische Tasks} \label{section:periodische_tasks}
			\subfile{sections/periodischeTasks.tex}
			\clearpage
		\subsection{\acf{udp}} \label{section:udp}
			\subfile{sections/udp.tex}
			\clearpage
		\subsection{\acf{fft}} \label{section:fft}
			\subfile{sections/fft.tex}
		\clearpage
		\subsection{Plattform} \label{section:plattform}
			\subfile{sections/Plattform.tex}
	\clearpage
	\section{Umsetzung des EDF-Schedulers} \label{section:der_edf_scheduler}
		\subfile{sections/Der_EDF_Scheduler.tex}
	\clearpage
	\section{Demo-Applikationen} \label{section:demo_applikationen}
		\subfile{sections/Demo_Applikationen.tex}
	\clearpage
	\section{Reflektion} \label{section:aufgetretene_probleme}
		\subfile{sections/Reflektion.tex}
	\clearpage
	\section{Fazit und Ausblick} \label{section:fazit_und_ausblick}
		\subfile{sections/Fazit_und_Ausblick.tex}
	\newpage
	\printbibliography[heading=bibnumbered, title=Literaturverzeichnis]
	\clearpage
	\section{Anlagen} \label{section:anlagen}
		\subfile{sections/Anlagen.tex}
\end{document}

