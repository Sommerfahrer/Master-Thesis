\documentclass[../EDF Master Thesis.tex]{subfiles}

\begin{document}
 Anwendungen auch Prozesse oder Tasks genannt, werden in Echtzeitsystemen oft periodisch, \ac{zb} das Abrufen eines Sensorwertes, ausgeführt.
 Periodische Tasks setzen sich aus folgenden Parametern zusammen:
 \begin{equ}[ht!]
    \begin{equation}
        T_i(\Phi_i, P_i, e_i, D_i)
    \end{equation}
    \begin{center}
        \begin{tabular}{lcr}
            $T_i$ & Task Nummer \\
            $\Phi_i$ & Phase \\
            $e_i$ & Ausführzeit (\ac{wcet}) \\
            $D_i$ & Deadline \\
        \end{tabular}
    \end{center}
    \caption{Parameter von periodischen Tasks}
    \label{form:parameter_of_periodic_tasks}
\end{equ}

Mit diesen vier Parameter jeder einzelnen periodischen Tasks können die verschiedenen Scheduler-Verfahren die Ausführung der einzelnen Tasks planen und umsetzen.
Im Zuge der Aufgabenstellung bei dieser Masterarbeit wurden keine Tasks mit einer Phasenverschiebung verlangt, daher kann der Parameter $\Phi_i = 0$ angesehen werden.
\end{document}


