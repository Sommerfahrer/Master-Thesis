\documentclass[../EDF Master Thesis.tex]{subfiles}

\begin{document}
 Anwendungen werden in Echtzeitsystemen oft periodisch ausgeführt.
 Periodische Tasks setzen sich aus folgenden Parametern zusammen:
 \begin{equ}[ht!]
    \begin{equation}
        T_i(t_{\phi,i}, t_{p,i}, t_{e,i}, t_{d,i})
    \end{equation}
    \begin{center}
        \begin{tabular}{lcr}
            $T$ & Task \\
            $t_\phi$ & Phase \\
            $t_e$ & Ausführungszeit (\ac{wcet}) \\
            $t_d$ & (relative) Deadline \\
        \end{tabular}
    \end{center}
    \caption[Parameter von periodischen Tasks]{Parameter von periodischen Tasks (\ac{iaa} \cite{echtzeit_systeme})}
    \label{form:parameter_of_periodic_tasks}
\end{equ}

Mit den in der \autoref{form:parameter_of_periodic_tasks} genannten Parametern jeder einzelnen periodischen Task können die verschiedenen Scheduler-Verfahren die Ausführung der einzelnen Tasks planen und umsetzen.
Im Zuge der Aufgabenstellung bei dieser Masterarbeit wurden keine Tasks mit einer Phasenverschiebung verlangt, daher kann der Parameter $t_\phi$ als Null angesehen werden \autocite{echtzeit_systeme}.
\end{document}


