\documentclass[../EDF Master Thesis.tex]{subfiles}

\begin{document}
Als Scheduler ( zu Deutsch: Steuerprogramm) wird eine Logik bezeichnet, welche die zeitliche Ausführung von mehreren Prozessen steuert.
Diese Logik wird als präemptiver oder kooperativer Scheduler in Betriebssystemen eingesetzt.
Der Scheduler ist für die zeitliche Anordnung der Prozesse verantwortlich.
Für die Neuanordnung, welche Task \ac{cpu}-Zeit bekommen soll, löst der Scheduler einen Context Switch, auch Taskwechsel genannt, aus.
Die \ac{cpu} ist für einen Scheduler die einzige Ressource von Relevanz, andere Hardware ist für einen Scheduler nicht relevant, wie z.B. \ac{rom} oder \ac{ram}.
Bei einem kooperativen Scheduling werden einem Prozess die benötigten Ressourcen übergeben und der Scheduler wartet, bis der Prozess vollständig abgearbeitet wurde.
Im Gegensatz zu einem kooperativen Scheduler kann ein präemptiver Scheduler einem Prozess die Ressourcen vor der Fertigstellung entziehen, um zwischenzeitlich diese anderen Prozessen zuzuweisen \parencite{mikrocontroller:001}.

Allgemein lassen sich die unterschiedliche Scheduler-Systeme in drei Rubriken unterteilen \parencite{wiki:007}:
\begin{itemize}
    \item \textbf{Stapelverarbeitungssysteme}: Zur Anfangszeit der Computer wurden Prozesse von Lochkarten eingelesen und nacheinander abgearbeitet und die Ergebnisse auf einem Magnetband abgespeichert oder ausgedruckt.
                                               Dabei bestanden Stapelverarbeitungssysteme oftmals aus mehreren Computern mit unterschiedler Rechenleistung, wobei die preiswerteren Computer mit weniger Rechenleistung Aufgaben wie das Lesen und Speichern übernahmen und die rechenstärkeren Computer die Abarbeitung des Programms vornahmen \autocite{grundkurs_betriebssysteme}.
    \item \textbf{Interaktive Systeme}: Eingaben von Benutzern sollten schnellstmöglich abgearbeitet werden, weniger wichtige Aufgaben wie z.B. die Aktualisierung der Uhrzeit werden unterbrochen oder erst im Nachhinein abgearbeitet \autocite{mikrocontroller:002}.
    \item \textbf{Echtzeitsysteme}: Das Echtzeitsystem muss die Fertigstellung des Prozesses innerhalb einer definierten Zeitspanne garantieren, sofern die \ac{cpu} Auslastung nicht über 100\% beträgt \autocite{mikrocontroller:002}.
\end{itemize}

Für die Anordnung der Prozesse selbst, wann welcher Prozess ausgeführt wird, gibt es verschiedene Arten von Schedulern.
In den nachfolgend vorgestellten Scheduler-Verfahren wird bei der Berechnung der Auslastung die Laufzeiten von Taskwechsel und Interrupts vernachlässigt.

\subsubsection{\acf{rms}} \label{section:rms}
    \ac{rms} ist ein präemptives Scheduling-Verfahren, welches die Prioritäten einzelner Prozesse nach deren Periodenlänge statisch anordnet.
    Je niedriger die Periode eines Prozesses ist, desto höher ist dessen Priorität.
    Mithilfe von \autoref{form:rms_CPU_utilization} kann die maximale Menge an Prozessen, die mit \ac{rms} verarbeitet werden können, berechnet werden.
    Durch eine zunehmende Anzahl von Prozessen nähert sich die Grenze dem Wert von $ln(2) \approx 0.693$, welches eine maximale Auslastung von 69,3\% bedeutet.
    Die maximale Auslastung bezieht sich dabei auf die \ac{cpu}-Auslastung, alle anderen Ressourcen wie z.B. Arbeitsspeicher werden als unbegrenzt angesehen.
    Sofern die maximale Auslastung nicht übertreten wird, verpasst kein Prozess die dazugehörige Deadline \parencite{echtzeit-grundlagen}.

    \begin{equ}[ht!]
        \begin{equation}
            U = \sum\limits_{i=1}^{n}{\frac{\Delta e_i}{\Delta d_i}} \le n \cdot \left(\sqrt[n]{2} -1\right)
        \end{equation}
        \begin{center}
            \begin{tabular}{lcr}
                $U$ & \ac{cpu}-Auslastung \\
                $\Delta e_i$ & Zeitspanne \\
                $\Delta d_i$ & Deadline \\
                $n$ & Anzahl der Prozesse \\
            \end{tabular}
        \end{center}
        \caption[\ac{rms} Berechnung der \ac{cpu}-Auslastung]{\ac{rms} Berechnung der \ac{cpu}-Auslastung \parencite{echtzeit-grundlagen}}
        \label{form:rms_CPU_utilization}
    \end{equ}
\subsubsection{\acf{dms}} \label{section:dms}
    Analog zu \ac{rms} arbeitet \ac{dms} präemptiv prioritätsbasiert, wobei bei \ac{dms} der Prozess mit der kürzesten Deadline die höchste Priorität bekommt. 
    Falls bei der Abarbeitung des aktuellen Prozesses ein neuer Prozess mit einer höheren Priorität eintrifft, wird der aktuelle Prozess unterbrochen und der Prozess mit der höheren Priorität bekommt die Rechenzeit.
    Die Berechnung der maximalen Auslastung erfolgt auch nach \autoref{form:rms_CPU_utilization}, dies führt zur gleichen maximalen Auslastung von 69,3\%, bei der keine Deadline überschritten wird \parencite{echtzeit-grundlagen}.

\clearpage
\subsubsection{\acf{edf}} \label{section:edf}
    Bei \ac{edf} bekommt die Task mit der am naheliegendsten Deadline \ac{cpu}-Zeit zugeteilt.
    Im Unterschied zu \ac{dms} wird bei \ac{edf} nur am Anfang oder Ende einer Task, die \ac{cpu}-Zeit zugeteilt bekommen hat, die Anordnung der Tasks aktualisiert.
    Dadurch wird eine Task, die einmal \ac{cpu}-Zeit zugeteilt bekommen hat, niemals unterbrochen, auch wenn die Deadline einer neu eingetroffen Task eine näherliegende Deadline besitzt.
    Auch ist die Anordnung, wie bei \ac{rms} und \ac{dms} nicht statisch durch fest gesetzte Prioritäten, sondern wird dynamisch am Anfang oder Ende jeder Task neu anhand der Deadlines angeordnet.
    Ein Context Switch, was ein Wechsel von einem Prozess zu einem anderen Prozess entspricht, findet nur vor Beginn eines Prozesses oder am Ende eines Prozesses statt.
    In Zuge dieser Masterarbeit wird ein Context Switch nur nach Beendigung eines Prozesses ausgelöst, sofern ein anderer Prozess eine höhere Priorität erhalten hat.
    Im Gegensatz zu \ac{rms} und \ac{dms} kann bei \ac{edf} der Scheduler die \ac{cpu} bis auf 100\% auslasten, dabei darf die Zeitspanne der einzelnen Prozesse bis zur Deadline nur jeweils größer oder gleich der Periode des Prozesses sein \parencite{echtzeit-grundlagen}.

    \begin{equ}[ht!]
        \begin{equation}
            U = \sum_{i=1}^{n} \frac{\Delta e_i}{\Delta d_i}
        \end{equation}
        \begin{center}
            \begin{tabular}{lcr}
                $U$ & \ac{cpu}-Auslastung \\
                $\Delta e_i$ & Zeitspanne \\
                $\Delta d_i$ & Deadline \\
                $n$ & Anzahl der Prozesse \\
            \end{tabular}
        \end{center}
        \caption[Berechnung der \ac{edf} \ac{cpu}-Auslastung]{Berechnung der \ac{edf} \ac{cpu}-Auslastung \parencite{echtzeit-grundlagen}}
        \label{form:edf_CPU_utilization}
    \end{equ}
\clearpage
\end{document}


