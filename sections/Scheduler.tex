\documentclass[../EDF Master Thesis.tex]{subfiles}

\begin{document}
Ein Scheduler regelt die zeitliche Ausführung von Prozessen des Betriebsystems als auch von Anwendungen.
Dabei sortiert der der Scheduler die einzelnen Prozesse nach einem bestimmten Verfahren, wobei alle Verfahren in zwei Rubriken unterteilt werden können.
\begin{itemize}
    \item Kooperatives Scheduling: Übergibt den Prozess der CPU und wartet bis der Prozess abgeschlossen.
    \item Präemptives Scheduling: Kann dem Prozess die CPU vor fertigstellung wieder entziehen, falls ein höher priorisierte Prozess zwischenzeitlich bearbeitet werden sollte.
\end{itemize}
Allgemein lassen sich die unterschiedliche Scheduler-Systeme in drei Rubriken unterteilen:
\begin{itemize}
    \item Stapelverarbeitungssysteme: Ankommende Prozesse werden nach Eingangszeit in eine Reihe (Queue) angeordnet, die dann nacheinander abgearbeitet wird.
    \item Interaktive Systeme: Eingaben von Benutzen sollten schnellstmöglich abgearbeitet werden, weniger wichtige Aufgaben wie \ac{zb} die Aktualisierung der Uhrzeit werden unterbrochen oder erst im Nachhinein abgearbeitet.
    \item Echtzeitsysteme: Das Echtzeitsystem garantiert die Fertigstellung des Prozesses innerhalb einer definierten Zeitspanne, sofern die CPU Auslastung nicht über 100\% beträgt.
\end{itemize}
Für die Anordnung selbst, wann welcher Prozess ausgeführt wird, gibt es verschiedene Arten von Scheduler.


\subsubsection{\ac{rms}}
    RMS ist ein Scheduling-Verfahren, welches die Prioritäten einzelner Prozesse nach der Periodendauer statisch anordnet.
    Je niedriger die Periode eines Prozesses ist, je höher ist dessen Priorität. \parencite{wiki:004}
    % \begin{equ}[ht!]
    %     \begin{equation}
    %         \sum\limits_{i=1}^{n}{\frac{C_i}{T_i}} \le n \cdot \left(\sqrt[n]{2} -1\right)
    %         \label{form: \ac{rms} Scheduling Grenzen}
    %         % \caption{\parencite{wiki:004}}
    %     \end{equation}
    % \end{equ}


\end{document}
