\documentclass[../EDF Master Thesis.tex]{subfiles}

\begin{document}
Als Scheduler (zu Deutsch Steuerprogramm) wird eine Logik bezeichnet, welche die zeitliche Ausführung von mehreren Prozessen steuert und wird als präemptiver oder kooperativer Scheduler in Betriebssystemen eingesetzt.
Bei einem kooperativen Scheduling wird einem Prozess die benötigten Ressourcen übergeben und wartet, bis der Prozess vollständig abgearbeitet wurde.
Im Gegensatz zu einem kooperativen Scheduler kann ein präemptiven Scheduler ein Prozess die Ressourcen vor der Fertigstellung entziehen, um zwischenzeitlich diese anderen Prozessen zuzuweisen \parencite{wiki:003}.

Allgemein lassen sich die unterschiedliche Scheduler-Systeme in drei Rubriken unterteilen \parencite{wiki:007}:
\begin{itemize}
    \item \textbf{Stapelverarbeitungssysteme}: Ankommende Prozesse werden nach Eingangszeit in eine Reihe (Queue) angeordnet, die dann nacheinander abgearbeitet wird.
    \item \textbf{Interaktive Systeme}: Eingaben von Benutzern sollten schnellstmöglich abgearbeitet werden, weniger wichtige Aufgaben wie \ac{zb} die Aktualisierung der Uhrzeit werden unterbrochen oder erst im Nachhinein abgearbeitet.
    \item \textbf{Echtzeitsysteme}: Das Echtzeitsystem garantiert die Fertigstellung des Prozesses innerhalb einer definierten Zeitspanne, sofern die \ac{cpu} Auslastung nicht über 100\% beträgt.
\end{itemize}
Für die Anordnung der Prozesse selbst, wann welcher Prozess ausgeführt wird, gibt es verschiedene Arten von Scheduler.


\subsubsection{\ac{rms}} \label{section:rms}
    \ac{rms} ist ein präemptives Scheduling-Verfahren, welches die Prioritäten einzelner Prozesse nach deren Periodenlänge statisch anordnet.
    Je niedriger die Periode eines Prozesses ist, je höher ist dessen Priorität.
    Mit Hilfe der Gleichung \ref{form:rms_CPU_utilization} kann die maximale Menge an Jobs, die mit \ac{rms} verarbeitet werden können, berechnet werden.
    Durch eine zunehmende Anzahl von Prozessen nähert sich die Grenze dem Wert von $ln(2) \approx 0.693$, welches eine maximale Auslastung von 69,3\% bedeutet.
    Die maximale Auslastung bezieht sich dabei auf die \ac{cpu}-Auslastung, alle anderen Ressourcen wie \ac{zb} Arbeitsspeicher werden als unbegrenzt angesehen.
    Sofern die maximale Auslastung nicht übertreten wird, verpasst kein Prozess die dazugehörige Deadline \parencite{wiki:004}.

    \begin{equ}[ht!]
        \begin{equation}
            U = \sum\limits_{i=1}^{n}{\frac{C_i}{T_i}} \le n \cdot \left(\sqrt[n]{2} -1\right)
        \end{equation}
        \begin{center}
            \begin{tabular}{lcr}
                $U$ & \ac{cpu}-Auslastung \\
                $C_i$ & Ausführungszeiten \\
                $T_i$ & Periodenlänge \\
                $n$ & Anzahl der Jobs \\
            \end{tabular}
        \end{center}
        \caption{\ac{rms} Berechnung der \ac{cpu}-Auslastung \parencite{wiki:004}}
        \label{form:rms_CPU_utilization}
    \end{equ}
\subsubsection{\ac{dms}} \label{section:dms}
    Analog zu \ac{rms} arbeitet \ac{dms} präemptiv prioritätsbasiert, wobei bei \ac{dms} der Prozess mit der kürzesten Deadline die höchste Priorität bekommt. 
    Falls bei der Abarbeitung des aktuellen Prozesses ein neuer Prozess mit einer höheren Priorität eintrifft, wird der aktuelle Prozess unterbrochen und der Prozess mit der höheren Priorität bekommt die Rechenzeit.
    Die Berechnung der maximalen Auslastung erfolgt mit der gleichen Formel \ref{form:rms_CPU_utilization}, dies führt zur gleichen maximalen Auslastung von 69,3\%, bei der keine Deadline überschritten werden \parencite{wiki:006}.
\subsubsection{\ac{edf}} \label{section:edf}
    Im Unterschied zu \ac{dms} wird bei \ac{edf} beim Eintreffen eines Prozesses mit höherer Priorität der aktuelle Prozess mit niedriger Priorität nicht unterbrochen, auch sind die Prioritäten nicht statisch, sondern werden bei jedem Scheduling-Vorgang neu vergeben.
    Ein Context Switch, das heißt ein Wechsel zwischen den Prozessen, findet nur vor Beginn eines Prozesses oder am Ende eines Prozesses statt.
    In Zuge dieser Masterarbeit wird ein Context Switch nur nach Beendigung eines Prozesses ausgelöst, sofern ein anderer Prozess eine höhere Priorität erhalten hat.
    Im Gegensatz zu \ac{rms} und \ac{dms} kann bei \ac{edf} der Scheduler die \ac{cpu} bis auf 100\% auslasten, dabei darf die Zeitspanne der einzelnen Prozesse bis zur Deadline nur jeweils größer oder gleich der Periode des Prozesses sein \parencite{wiki:006}.

    \begin{equ}[ht!]
        \begin{equation}
            U = \sum_{i=1}^{n} \frac{C_i}{T_i}
        \end{equation}
        \begin{center}
            \begin{tabular}{lcr}
                $U$ & \ac{cpu}-Auslastung \\
                $C_i$ & Ausführungszeiten \\
                $T_i$ & Periodenlänge \\
                $n$ & Anzahl der Jobs \\
            \end{tabular}
        \end{center}
        \caption{\ac{edf} Berechnung \ac{cpu}-Auslastung \parencite{wiki:006}}
        \label{form:edf_CPU_utilization}
    \end{equ}
\end{document}


