\documentclass[../EDF Master Thesis.tex]{subfiles}

\begin{document}
    Ein Teil der Aufgabenstellung beinhaltet das Erstellen von zwei Demo Applikationen, welche die Funktion des \ac{edf} Schedulers unter Beweis stellen.

    \subsection{Blinky Demo} \label{section:blinky_demo}
        Für eine schnelle optische Vorführung des \ac{edf} Schedulers wurde eine 'Blinky Demo' erstellt, welche drei \ac{edf} Tasks beinhaltet.
        Jede der drei Tasks toggelt eine unterschiedliche \ac{led} besitzen jedoch die gleichen \ac{edf} Parameter, welche in \autoref{table:blinky_demo_task_parameter} dargestellt sind.
        Die drei Tasks wurden exakt so gewählt, dass Sie die verschiedenen möglichen Status des \ac{edf} Schedulers simuliert und optisch dargestellt werden.
        Der Benutzer kann mit Hilfe des integrierten Button auf dem STM32f769I-Disc0 Board die drei Tasks, wie in \autoref{fig:blinky_demo_ablaufdiagramm} zu sehen, erstellen und löschen.
        
        \begin{figure}[ht!]
            \begin{center}
                \begin{tikzpicture}[node distance = 2cm, auto]
                    % \draw[very thin,lightgray](-2,-9)grid[step=.5cm]+(6,10);
                
                    % Place nodes
                    \node [papProcess, label={[papLabel]right:Button Counter = 0}](pro1){Warten auf Benutzer};
                    \node [papProcess, below = of pro1,label={[papLabel]right:Button Counter = 0}](pro2){Idle Task};
                    \node [papProcess, below = of pro2,label={[papLabel]right:Button Counter = 1}](pro3){Task 1 + Idle Task};
                    \node [papProcess, below = of pro3,label={[papLabel]right:Button Counter = 2}](pro4){Task 1 + Task 2};
                    \node [papProcess, below = of pro4,label={[papLabel]right:Button Counter = 3}](pro5){Task 1 + Task 2 + Task 3};
                    \node [papProcess, below = of pro5,label={[papLabel]right:Button Counter = 4}](pro6){Task 2 + Task 3};
                    \node [papProcess, below = of pro6,label={[papLabel]right:Button Counter = 5}](pro7){Task 3 + Idle Task};
                            % invisible node helpful later
                    \node[left=1cm of pro6,scale=0.05](inv){};
                
                    % Draw edges
                    \path [line] (pro1) -- (pro2);
                    \path [line] (pro2) -- (pro3);
                    \path [line] (pro3) -- (pro4);
                    \path [line] (pro4) -- (pro5);
                    \path [line] (pro6) -- (pro7);
                    \path[-,draw] (pro7) -| node{} (inv.north);
                    \path[line]{} (inv.north) |- node[above]{reset} (pro1);
                \end{tikzpicture}  
            \end{center}
            \caption{Blinky Demo Ablaufdiagramm}
            \label{fig:blinky_demo_ablaufdiagramm}
        \end{figure}


        \begin{table}[ht!]
            \centering
            \begin{tabular}{l|c|c|c}
                Tasks & Periode & Deadline & \ac{wcet} \\
                \hline
                Task 1 & 50 & 100 & 100\\
                Task 2 & 50 & 100 & 100\\
                Task 3 & 50 & 100 & 100\\
            \end{tabular}
            \caption{Blinky Demo Task Parameter}
            \label{table:blinky_demo_task_parameter}
        \end{table}

    
    \subsection{\ac{fft} Demo} \label{section:fft_demo}


\end{document}
