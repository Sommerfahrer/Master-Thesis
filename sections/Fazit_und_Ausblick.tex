\documentclass[../EDF Master Thesis.tex]{subfiles}

\begin{document}
    Als Ergebnis dieser Thesis wurde ein \ac{edf} Scheduler in \ac{freertos} implementiert, welcher einen geringen Overhead sowie Footprint besitzt.
    Dabei wurde darauf geachtet, dass keine Funktionen oder Variablen der \ac{freertos} Installation modifiziert wurden, um eine einfache Portierbarkeit auf andere Mikrocontroller zu ermöglichen.
    Die Entscheidung den \ac{freertos} SysTick zu verwenden, trägt weiter dazu bei.
    Ebenfalls führt die Implementierung eines Debug Modes und deren Kommunikation mit dem \ac{usart} zu einer einfachen Fehlerbehebung während eines Entwicklungsprozesses.
    Im weiteren Verlauf wurde eine 'Blinky Demo' erstellt, welche die verschiedenen Status des \ac{edf} Schedulers anhand von \ac{led}s darstellen kann.
    Des Weiteren konnte mit der '\ac{fft} Demo' auch ein wesentlich komplexeren Anwendungsfall für die Überprüfung des \ac{edf} Schedulers implementiert werden.
    Die erstellte \ac{gui} erleichtert dabei die generierung verschiedenster Signale, sowie auch das darstellen des generierten Signals und der \ac{fft} Ergebnisse.
    Somit konnte die korrekte Funktion des \ac{edf} Schedulers mit zwei Demo Applikationen unter Beweis gestellt werden.

    Die Implementierung eines \ac{edf} Schedulers in \ac{freertos} ermöglicht eine größere Auswahl des Scheduler-Betriebes, wodurch für jede Applikation die Möglichkeit steigt, den passenden Scheduler verwenden zu können.
    In Zukunft könnten noch weitere Scheduler-Arten implementiert werden, welche den Einsatzzweck der Arbeit noch erweitern würde und somit attraktiver für die Weiterentwicklung dieses Projektes sorgen würde. 
    
    


\end{document}
