\documentclass[../EDF Master Thesis.tex]{subfiles}

\begin{document}
\ac{freertos} ist ein Echtzeitbetriebssystem für eingebettete Systeme, welches unter der freizügigen Open-Source Lizenz \href{https://de.wikipedia.org/wiki/MIT-Lizenz}{MIT} steht.
Für eine leichte Wartbarkeit wurde \ac{freertos} weitestgehend in C entwickelt, außerdem ist der Scheduler des Betriebsystems zwischen präemptiver und kooperativer Betrieb konfigurierbar um verschiedene Einsatzzwecke abzudecken \parencite{wiki:002}.
Des Weiteren wurde \ac{freertos} auf über 40 Mikrocontroller-Architekturen portiert \parencite{freertos}, um eine größere Bandbreite zu erreichen.
\ac{freertos} hat wenig Overhead und der Kernel unterstützt Multithreading, Warteschlangen, Semaphore, Software-Timer, Mutexes und Eventgruppen \parencite{freertos-features}.




\begin{lstlisting}[language=C, caption=FreeRTOS Task creation and deletion]
    createEDFTask(              // Task Creation
        testFunc,               // Pointer to task entry function
        "Test Function",        // A descriptive name for the task
        300,                    // The number of words to allocate
        NULL,                   // A value that will be passed into the task 
        1,                      // WCET in ms
        5,                      // Period of Task in ms
        4 );                    // Deadline of Task in ms
    \end{lstlisting}

\end{document}
