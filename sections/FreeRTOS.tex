\documentclass[../EDF Master Thesis.tex]{subfiles}

\begin{document}
    \ac{freertos} ist ein Echtzeitbetriebssystem für eingebettete Systeme, welches unter der freizügigen Open-Source Lizenz \href{https://de.wikipedia.org/wiki/MIT-Lizenz}{MIT} steht.
    Für eine leichte Wartbarkeit wurde \ac{freertos} weitestgehend in C entwickelt, außerdem ist der Scheduler des Betriebsystems zwischen präemptiver und kooperativer Betrieb konfigurierbar um verschiedene Einsatzzwecke abzudecken \parencite{wiki:002}.
    Des Weiteren wurde \ac{freertos} auf über 40 Mikrocontroller-Architekturen portiert \parencite{freertos}, um eine größere Bandbreite zu erreichen.
    Als Scheduler (zu Deutsch Steuerprogramm) wird eine Logik bezeichnet, welche die zeitliche Ausführung von mehreren Prozessen steuert und wird als präemptiver oder kooperativer Scheduler in Betriebssystemen eingesetzt.
    Bei einem kooperativen Scheduling wird einem Prozess die benötigten Ressourcen übergeben und wartet, bis der Prozess vollständig abgearbeitet wurde.
    Im Gegensatz zu einem kooperativen Scheduler kann ein präemptiven Scheduler ein Prozess die Ressourcen vor der Fertigstellung entziehen, um zwischenzeitlich diese anderen Prozessen zuzuweisen \parencite{wiki:003}.
    \ac{freertos} hat wenig Overhead und der Kernel unterstützt Multithreading, Warteschlangen, Semaphore, Software-Timer, Mutexes und Eventgruppen \parencite{freertos-features}.
\end{document}
