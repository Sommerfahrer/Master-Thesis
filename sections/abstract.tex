\documentclass[../ImageClassifier.tex]{subfiles}

\begin{document}
    Without any or enough training data there is no way for deep learning to make a robust classification.
    In this report, google images are used for training and validated with several datasets and classes.
    In addition, there is a \ac{grad-cam} (section \ref{chap:grad-cam}) implemented for localization of the predicted classes.
    The main goal of the project was to find out whether sufficient accuracy for image classification can be achieved with image search engines like google and bing images.
    Images from search engines can also contain other classes, therefore a threshold was used to get only the most reliable predictions of the network.
    For easier usability, a complete toolchain is implemented, the user only needs to type in the required classes, the path to the images to be predicted and optional the desired threshold value.
    The program automatically searches locally for pre-trained networks, if this is not the case the program loads about 600 pictures for every class and starts to learn the new class with transfer learning.
    After transfer learning, the network will be fine tuned for a few epochs.
    Finally the newly trained network will start to predict all images in the given path.
    The results can be used for pre-labelling as well as a more precisely dataset for a different network training.
\end{document}
