\documentclass[../EDF Master Thesis.tex]{subfiles}

\begin{document}
    Bei der Entwicklung des \ac{edf}-Schedulers traten folgende Probleme auf:
    \begin{itemize}
        \item \textbf{Natives \ac{freertos}}: Als \ac{ide} wurde in dieser Arbeit aufgrund der einfachen Bereitstellung der Board-Treiber, die Software 'STM32CubeIDE' von ST verwendet.
            Bei der Konfiguration der Hardware und Software kann eine \ac{freertos} Installation ebenfalls ausgewählt werden.
            Leider stellte sich heraus, dass dabei ein Wrapping Layer automatisch mit konfiguriert und implementiert wird, um einen einfachen Wechsel zwischen den unterschiedlichen \ac{rtos} Betriebssystemen zu ermöglichen.
            Dieser Wrapping Layer verhindert aber eine einfache Portierung auf andere Mikrocontroller, speziell Mikrocontroller die nicht von dem Hersteller ST stammen.
            Resultierend aus diesen Erfahrungen wurde dann ein natives \ac{freertos} ohne Wrapping Layer in dieser Arbeit auf dem STM32F769I-Disc0 implementiert.
        \item \textbf{\ac{freertos}+TCP}: Bei der Integration des \ac{freertos}+TCP Stack konnte lange nicht die Ethernet-Schnittstelle aktiviert werden.
            Schließlich stellte sich heraus, dass bei der Konfiguration der Netzwerkschnittstelle in 'STM32CubeIDE' automatisch eine Initialisierungsfunktion integriert und diese in der Main-Funktion aufgerufen wird.
            Das Problem bestand darin, dass der \ac{freertos}+TCP Stack eine eigene Initialisierungsfunktion verlangt und die Schnittstelle daher blockiert wurde.
            Nachdem die 'Standard'-Initialisierungsfunktion entfernt wurde, konnte die Netzwerkschnittstelle aktiviert werden.
        \item \textbf{Ethernetschnittstelle}: Für die Kommunikation zwischen Host-Computer und dem Mikrocontroller wurde das \ac{udp}-Protokoll und somit die Netzwerkschnittstelle des Mikrocontrollers ausgewählt.
            Damit nicht zwischen dem Entwickeln und der Recherche im Internet die RJ45-Kabel am Host-Computer gewechselt werden musste, wurde anfangs der Mikrocontroller direkt an einem Switch im heimischen Netzwerk angesteckt.
            Nachdem die Integration des \ac{freertos}+TCP Stacks abgeschlossen war, wurden mehrere Ping Tests durchgeführt, doch der Mikrocontroller blieb nicht erreichbar.
            Nach der Überprüfung der Konfiguration wurde der Mikrocontroller direkt mit dem Host-Computer verbunden und antwortete sofort auf die Ping-Anfragen.
            Wahrscheinlich wurden in dem vorhandenen Netzwerk für den Mikrocontroller zu viele Broadcastanfragen gesendet.
        \item \textbf{\ac{fft}-Größe}: Je größer die \ac{fft} konfiguriert wird, desto länger braucht deren Berechnung.
            Gleichzeitig sendet der Host-Computer weiterhin neue Pakte zum Mikrocontroller, was dazu führt, dass bei der Erhöhung der \ac{fft}-Größe auch die Stackgröße der udpReceivingTask vergrößert werden musste.
            Ansonsten kommt es zu einem Buffer Overflow.
            Das verwendete Board hat eine Arbeitsspeichergröße von 512kb und somit ausreichend Platz für größere Variablen.
            Allerdings wurde es als nicht sinnvoll erachtet, den Hauptteil des Arbeitsspeichers mit Buffer Variablen zu belegen.
    \end{itemize}
\end{document}
