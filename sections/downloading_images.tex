\documentclass[../ImageClassifier.tex]{subfiles}

\begin{document}
    In order to download images by search engines like in this project with Google Images and Bing, there are several options.
    The simplest option would be to download images of the user given class by hand, but this is not practical.
    There is a big problem with scraping search enginges.
    They do not want to be read out automatically, because that means unnecessary server load without advertising.
    The result of this is, that they have costs but no earnings.
    To prevent to be read out automatically they often change their \ac{html} tags.
    This means it`s a cat and mouse game.
    During this project, the first library \flqq Google Images Download\flqq\,\parencite{google-image-download} downloading Google and Bing Images was deprecated for Google Images in the middle of June because they changed their \ac{html} tags.
    An alternative scraping library was found in \flqq Image-Downloader\flqq\,\parencite{image-downloader}.
    With this library it was possible to download up to 800 Google Images and 1000 Bing Images per keyword.
\end{document}
