\documentclass[../EDF Master Thesis.tex]{subfiles}

\begin{document}

Der \ac{edf}-Scheduler beinhaltet bis dato kein eigenes Repository sondern wurde zu den zwei Demo Applikationen hinzugefügt.
Alle erstellten Projekte sind online unter folgenden Links zu erreichen:

\begin{itemize}
    \item \textbf{\ac{fft} Demo}: https://github.com/punicawaikiki/FreeRTOS-Ethernet-EDF
    \item \textbf{Blinky Demo}: https://github.com/punicawaikiki/EDF-Blinky-Test
    \item \textbf{\ac{gui}}: https://github.com/punicawaikiki/EDF-Python-Interface
\end{itemize}

\subsection{Verzeichnis des Datenträgers}

\begin{itemize}
    \item Daten\-CD\_Masterthesis
    \begin{itemize}
        \item 01\_Masterthesis.pdf
        \item 02\_FFT\_Demo
        \item 03\_Blinky\_Demo
        \item 04\_GUI
    \end{itemize}
\end{itemize}

\textbf{Anmerkungen:}\\
Die Binaries der zwei Demo-Applikationen befinden sich unter dem Debug-Ordner in den Ordnern '02\_FFT\_Demo' und '04\_Blinky\_Demo'.
Für die Installation auf dem STM32F769I-Disc0 Board muss lediglich die Binary-Datei in das Verzeichnis des Mikrocontrollers kopiert werden.
Bei der '02\_FFT\_Demo' muss der Mikrocontroller zusätzlich mit dem Host-Computer verbunden werden.
Dabei sollte der Host-Computer für die Kommunikation mit dem Mikrocontroller die IP-Adresse 192.168.1.1 besitzen.
Um die \ac{gui} zu starten, muss die 'gui.exe' unter '03\_FFT\_GUI/dist' gestartet werden.
Sollte die Bedienung der \ac{gui} auf einem anderem Betriebssystem als auf Windows ausgeführt werden, muss die Python-Datei 'gui.py' mit den installierten Modulen aus der Datei 'requirements.txt' gestartet werden.


\end{document}
