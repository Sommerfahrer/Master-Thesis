\documentclass[../EDF Master Thesis.tex]{subfiles}

\begin{document}

Der \ac{edf} Scheduler lässt sich mit einer einzelnen C-Datei, sowie einer Header-Datei in \ac{freertos} implementieren.
Dies vereinfacht die Implementierung auf andere \ac{freertos} Versionen, sowie auch das portieren auf andere Plattformen.
Um die Portabilität weiter zu vereinfachen, wurde der \ac{freertos} SystemTick (SysTick) mit der standartmäßigen Genauigkeit von einer Millisekunde und nicht ein weiterer benutzerdefinierter Timer, wie es die Cortex-M Serie ermöglicht, benutzt.
Aus dieser Entscheidungen resultiert auch die Genauigkeit des Schedulers von einer Millisekunde, wobei auftauchende Interrupts, wie die \ac{cpu} Zeit des SysTick, in dieser Arbeit vernachlässigt werden.
Die Periode des SysTick lässt sich unter der Datei 'FreeRTOSConfig.h' , wie in der \autoref{section:freertos_sys_tick} einstellen, jedoch ändert sich dabei auch die Genauigkeit des \ac{edf} Schedulers.
Als Standart ist bei einer \ac{freertos} Installation eine Millisekunde definiert, welche für den \ac{edf} Scheduler einen guten Kompromiss zwischen Applikationslaufzeit und Genauigkeit bietet.

\begin{lstlisting}[language=C, caption=FreeRTOS Idle Task Hook, label=code:freertos_idle_task_hook]
    #define configTICK_RATE_HZ						( 1000 ) // 1ms SysTick
\end{lstlisting}

\subsection{Implementierung} \label{section:implementierung}
    Um eine einfache und zugleich sehr performante Implementierung eines \ac{edf} Schedulers in \ac{freertos} zu gewährleisten, wurde in dieser Arbeit der \ac{freertos} Scheduler weiterverwendet.
    Der \ac{freertos} Scheduler arbeitet, wie in \autoref{section:freertos_scheduling_richtlinie} bereits behandelt, prioritätsbasiert.
    Mit dem Aufruf der Funktion 'rescheduleEDF' am Ende jeder \ac{edf} Task, wird zwischen allen erstellten \ac{edf} Tasks, die Priorität der nächsten Task nachdem \ac{edf}-Prinzip auf die höchste Priorität gesetzt und die aktive Task auf eine niedere Priorität als die \ac{freertos} Idle Task gesetzt.
    Anschließend wird ein Context Switch ausgelöst und die nach dem \ac{edf}-Prinzip nächste Task wird mit \ac{freertos} Scheduler aufgerufen und ausgeführt.
    Falls alle \ac{edf} Tasks bereits in Ihrer Periode ausgeführt wurden und alle Deadlines der Tasks weiter als ihre Periode in der Zukunft liegen, wird die \ac{freertos} Idle Task aufgerufen.

    \begin{center}
        \begin{table}[ht!]
            \begin{tabular}{l|c}
                Tasks & Priorität \\
                \hline
                Deaktivierte \ac{edf} Task(s) & 0\\
                \ac{freertos} Idle Task & 1\\
                Auführende \ac{edf} Task & 2\\
                'rescheduleEDF' ausführende Task & 3
            \end{tabular}
            \caption{\ac{edf}-Prioritäten der Tasks}
            \label{table:edf_prioritaeten_der_tasks}
        \end{table}
    \end{center}

    Der \ac{edf} Scheduler arbeitet mit vier unterschiedlichen Prioritäten, wie in \autoref{table:edf_prioritaeten_der_tasks} dargestellt und inkrementiert bei der Implementierung die Idle Task von der niedrigsten Priorität '0' auf die zweit niedrigste Priorität '1' in der \ac{freertos} Umgebung.
    Alle \ac{edf} Tasks die nicht ausgeführt werden sollen, erhalten die Priorität 0, das führt dazu, dass die Idle Task immer ausgeführt wird und verhindert die nicht gewollte Ausführung einer \ac{edf} Task.
    Um ein \ac{edf} Scheduling in der Idle Task zu ermöglichen, wird die 'rescheduleEDF'-Funktion mit Hilfe des Idle Task Hook, wie in \autoref{code:freertos_idle_task_hook} beschrieben, aufgerufen.
    Die Task, welche aktuell ausgeführt wurde und nun am Ende die 'rescheduleEDF'-Funktion ausführt, erhält während der Funktionsausführung die höchste \ac{edf}-Priorität mit dem Wert '3', um einen vorzeitigen Context Switch zur nächsten, nach dem \ac{edf}-Prinzip ausgewählte Task, zu unterbinden.
    Nachdem alle \ac{edf}-Parameter von allen \ac{edf} Tasks neu gesetzt wurden, gibt die aktuelle Task die höchste Priorität mit dem '3' ab und erzwingt einen Context Switch zu der nächsten zu ausführenden \ac{edf} Task mit der Priorität '2'.
    


\subsection{Rescheduling Ablauf} \label{section:rescheduling_ablauf}

    Wie in \autoref{section:implementierung} erwähnt, wird mit Hilfe der Funktion 'rescheduleEDF' und dem unveränderten \ac{freertos} Scheduler der \ac{edf} Scheduler implementiert.


\subsection{Fehlerbehandlung} \label{section:Fehlerbehandlung}

    Sollten zwei \ac{edf} Tasks die gleichen Deadline besitzen, wird die Task, die bisher weniger Ausführungen vorweist, als nächste zu ausführende Task ausgesucht.
    Falls eine Task ihre Deadline überschritten hat oder die Zeit für die Ausführung nicht mehr reicht, werden die \ac{edf} Parameter angepasst.
    Dadurch wird keine Task bei der Fehlerbehandlung bevorzugt oder benachteiligt, wobei natürlich die Task mit der kürzesten Deadline und der kleinsten Periode eine höhere Wahrscheinlichkeit besitzt, ausgeführt zu werden. 

\subsection{Debug Mode} \label{section:debug_mode}

    
    



\end{document}
