\documentclass[../EDF Master Thesis.tex]{subfiles}

\begin{document}

Der \ac{edf} Scheduler lässt sich mit einer einzelnen C-Datei, sowie einer Header-Datei in \ac{freertos} implementieren.
Dies vereinfacht die Implementierung auf andere \ac{freertos} Versionen, sowie auch das portieren auf andere Plattformen.
Um die Portabilität weiter zu ermöglichen, wurde der \ac{freertos} SystemTick (SysTick) mit der standartmäßigen Genauigkeit von einer Millisekunde und nicht einen weiteren benutzerdefinierten Timer, wie es die Cortex-M Serie ermöglicht, benutzt.
Daher resultiert auch die Genauigkeit des Schedulers von einer Millisekunde, wobei auftauchende Interrupts vernachlässigt werden.
Die Periode des SysTick lässt sich unter der Datei 'FreeRTOSConfig.h' einstellen, jedoch ändert sich somit auch die Genauigkeit des \ac{edf} Schedulers.
Als Standart ist bei einer \ac{freertos} Installation eine Millisekunde definiert, welche für den \ac{edf} Scheduler einen guten Kompromiss zwischen Applikationslaufzeit und Genauigkeit bietet.

\subsection{Implementierung} \label{section:implementierung}
    Um eine einfache und zugleich sehr schnelle Implementierung eines \ac{edf} Schedulers in \ac{freertos} zu gewährleisten, wurde in dieser Arbeit der \ac{freertos} Scheduler weiterverwendet.
    Der \ac{freertos} Scheduler arbeitet, wie in \autoref{section:freertos_scheduling_richtlinie} bereits behandelt, prioritätsbasiert.
    Durch den aufruf einer in dieser Masterarbeit entworfene Funktion am Ende jeder \ac{edf} Task, wird zwischen allen erstellten \ac{edf} Tasks, die Priorität der nächsten Task nachdem \c{edf} Prinzip auf die höchste Priorität gesetzt und die aktive Task auf eine niedere Priorität als die \ac{freertos} Idle Task gesetzt.
    


Ein weiteres Feature des \ac{edf} Schedulers ist, dass durch die weiterverwendung des standart Scheduler von \ac{freertos}, dieser parallel verwendet werden kann, allerdings ist zu beachten, dass eine höhere priorisierte Task, als die des \ac{edf} Schedulers definierte höchste Task, die Deadlines der \ac{edf} Tasks zu überziehen.
Auf der anderen Seite wird eine niedrigere priorisierte Task, wie von dem \ac{edf} Scheduler ausgewählte Task, niemals ausgeführt wird.

Für die Implementierung des \ac{edf} Scheduler in \ac{freertos} wurde darauf geachtet, dass keine Programmabschnitte von \ac{freertos} selbst modifiziert wurden.
Dadurch ist es möglich, 

\end{document}
