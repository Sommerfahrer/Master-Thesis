\documentclass[../EDF Master Thesis.tex]{subfiles}

\begin{document}
    In den meisten \ac{rtos} werden fixed-priority pre-emptive Scheduler verwendet, bei der jeder Prozess eine feste Priorität zugewiesen bekommt, welche über die gesamte Laufzeit konstant bleibt.
    Vorteil dieser Scheduler Prinzips ist die weniger komplexe Implementierung im Kernel und der geringe Overhead.
    Die Unterstützung mehrerer Schedule-Verfahren im Kernel eines \ac{rtos} würde dessen Weiterentwicklung stark beeinflussen.
    Allerdings gibt es die Möglichkeit ein anderes Scheduling-Verfahren außerhalb des Kernels, also im Userspace eines \ac{rtos} zu implementieren.
    Bei einem Scheduling-Verfahren nach dem \ac{edf}-Prinzip kann die \ac{cpu} maximal ausgelastet werden, des Weiteren muss die Prozessstruktur und deren Prioritäten nicht vorher festgelegt werden.
    Aus diesen Gründen besteht das Ziel dieser Arbeit darin, einen \ac{edf}-Scheduler im Userspace von \ac{freertos} zu implementieren.
    Durch die Entwicklung zweier Demo-Applikationen soll die korrekte Funktion des Schedulers nachgewiesen werden.
    Dabei dient eine Demo als Vorführung des \ac{edf}-Schedulers, bei der alle Status innerhalb und außerhalb der Grenzen des \ac{edf}-Schedulers abgefragt werden.
    Die zweite Demo stellt eine komplexere Anwendung dar, wobei drei unterschiedliche Prozesse Daten verarbeiten und untereinander kommunizieren müssen. 
    Einleitend werden die Grundlagen für das Verständnis der Arbeit beschrieben, anschließend wird die Implementierung des \ac{edf}-Schedulers erläutert.
    Danach werden die zwei erarbeiten Demo-Applikation vorgestellt und die verschiedenen Status des Schedulers innerhalb der Applikationen.\\
    Der erarbeitete Scheduler soll in Zukunft als Erweiterung für \ac{freertos} dienen.
    Das trennen des Scheduler-Codes von \ac{freertos}, der geringe Overhead und Footprint, sowie die Verwendung des \ac{freertos} Timer ermöglicht eine hohe Portierbarkeit auf verschiedene Plattformen.
    Des Weiteren weisen die erarbeiteten Demo-Applikationen die korrekte Funktion des Schedulers nach.
\end{document}
