\documentclass[../ImageClassifier.tex]{subfiles}

\begin{document}
    At the beginning the user must start the programm with the terminal.
    The programm itself takes two mandatory arguments \flqq\textit{\texttt{-{}-}classes \textless keyword\_1,keyword\_2\textgreater\frqq}\,(in short: \textit{\flqq -c \textless keyword\_1,keyword\_2 \textgreater \frqq}) for the detecting classes and \flqq\textit{\texttt{-{}-}prediction\_path\\ \textless path\_to\_predicting\_images\textgreater \frqq}\,(in short: \textit{\flqq -p \textless path\_to\_predicting\_images\textgreater \frqq}) with the path to predicting images.
    Another optional argument is the threshold for the predicted images, which is very important for the accuracy of the results.
    The threshold has the default value 0.99, which means only results with 99\% certainty are saved in results, images below the threshold would be skipped.
    In order to the default value of the threshold you must use the argument \flqq \texttt{-{}-}threshold \textless threshold\_value\textgreater\frqq (in short: \flqq -t \textless threshold\_value\textgreater\frqq).
    In order to start the program with all arguments, the command must look like this:
    \begin{lstlisting}[language=python, title={Launch of the toolchain}]
        python classifier_training.py -c "cat,dog" -p "~/eiwomisau/test/mixed" -t 0.98\end{lstlisting}
    Now the toolchain is looking for an already trained network, if a suitable network is available, the network will be loaded and starts predicting the user given images.
    If there is no suitable network, the toolchain starts to download images with the desired keywords.
\end{document}
