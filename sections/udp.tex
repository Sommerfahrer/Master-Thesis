\documentclass[../EDF Master Thesis.tex]{subfiles}

\begin{document}
    Für die Erfüllung der Aufgabe dieser Masterarbeit musste ein Kommunikationsmittel zwischen Computer und Mikrocontroller ausgesucht werden, welches verschiedene Eigenschaften erfüllt:
    \begin{itemize}
        \item \textbf{verbindungslos}: Der Sender sendet Daten ohne eine Bestätigung des Eingangs der Daten zu bekommen.
                                       Dies hat den Vorteil einer latenzärmeren Datenübertragung.
        \item \textbf{wenig Overhead}: Das Empfangen und Senden sollte so wenig wie möglich Ressourcen, wie \ac{cpu}-Zeit, blockieren.
        \item \textbf{hohe Übertragungsrate}: Es sollten so viele Daten wie möglich, innerhalb einer bestimmten Zeit übertragen.
    \end{itemize}

    Aus diesen Gründen wurde das Netzwerkprotokoll \ac{udp}, welches heutzutage in vielen Anwendungen, wie \ac{zb} \ac{ip}-Telefonie und Videostreams, verwendet.
    Für den Verbindungsaufbau verwendet \ac{udp} Ports, welche Teil einer Netzwerk-Adresse darstellen, um versendete Daten dem gewünschten Programm an der Gegenstelle zukommen zu lassen \parencite{wiki:009}.
\end{document}
