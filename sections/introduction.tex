\documentclass[../ImageClassifier.tex]{subfiles}

\begin{document}
    One of the most time consuming, difficult and expensive thing on machine learning is to create a dataset.
    Why shouldn`t use \href{https://www.google.de/imghp?hl=de&tab=wi&authuser=0&ogbl}{Google} or \href{https://www.bing.com}{Bing} images to get a huge amount of images pre-labeled with the required classes?
    The answer is not that easy, for example if you are searching for a car within a search engine, you would get sets of images of the interior of a car as well as of the exterior.
    This is one argument against using search engines as data source, another argument is the handling with licenses of the images or to specific classes like not very well known things.
    Nevertheless to find out, how accurate and more or less useful it can be, to use search engines for datasets.
    The toolchain contains all parts of loading images, resize, reformat, transfer learning, fine tuning and predict the user given classes on images.
    In addition the tool can make also a \ac{grad-cam} for localize the classes within the images.
\end{document}
